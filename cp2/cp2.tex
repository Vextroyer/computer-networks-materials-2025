\documentclass[12pt]{amsart}

\addtolength{\hoffset}{-2.25cm}
\addtolength{\textwidth}{4.5cm}
\addtolength{\voffset}{-2.5cm}
\addtolength{\textheight}{5cm}
\setlength{\parskip}{0pt}
\setlength{\parindent}{15pt}

\usepackage{amsthm}
\usepackage{amsmath}
\usepackage{amssymb}
\usepackage[spanish]{babel}
\usepackage[colorlinks = true, linkcolor = black, citecolor = black, final]{hyperref}

\usepackage{graphicx}
\usepackage{multicol}
\usepackage{ marvosym }
\usepackage{wasysym}
\usepackage{tikz}
\usetikzlibrary{patterns}

\newcommand{\ds}{\displaystyle}
\newcommand{\horrule}[1]{\rule{\linewidth}{#1}}

\setlength{\parindent}{0in}

\pagestyle{empty}
\begin{document}
	\hrule
	\smallskip
	\begin{center}
		{\scshape {\large Redes de Computadoras} \\
			Curso 2025-2026} \\ \smallskip
		\textbf{Clase Práctica \# 2} \\
		{\small \textbf{Tema:} Simulacion de redes con Docker: Router y Hosts}
	\end{center}
	\vspace{-8px}
	\rule{\linewidth}{2pt}
	
	{\scshape Facultad de Matemática y Computación}  \hfill {\scshape Universidad de La Habana}
	
	\bigskip\bigskip
	
	
	
	\vspace{1cm}
	
	\begin{center}
		\textit{La única forma de hacer un gran trabajo es amar lo que haces.}
	\end{center}
	
	\begin{flushright}
		- Steve Jobs
	\end{flushright}
	
	\vspace{1cm}
	
	% Ejercicios
	\begin{enumerate}
		
		\thispagestyle{empty}
		
		%------------------------------------------------------------%
		\item \textbf{\textit{Carpeta 1: dockerfiles}}
		
		\medskip
		\noindent Contiene 3 archivos \textit{Dockerfile}
			
		\medskip
			\begin{itemize}
				\item \textbf{\textit{host1}} : Define el entorno del primer host
				\smallskip
				\item \textbf{\textit{host2}} : Define el entorno del segundo host
				\smallskip
				\item \textbf{\textit{router}}: Define el entorno del router (con herramientas de red como `ip', `ping', `iptables', etc.).
			\end{itemize}
		
		\bigskip\bigskip
		%------------------------------------------------------------%
		
		%------------------------------------------------------------%
		\item \textbf{\textit{Carpeta 2: images}}
		
		\medskip
		\noindent Contiene las imágenes Docker ya construidas (`.tar' o listas para cargar).
		
		\bigskip\bigskip
		%------------------------------------------------------------%
		
		%------------------------------------------------------------%
		\item \textbf{\textit{Carpeta 3: scripts}}
		
		\medskip
		\noindent Contiene cuatro scripts esenciales para la práctica: 

		\medskip
			\begin{itemize}
				\item \textbf{\textit{build\_images.sh}} : Construye las imágenes desde los Dockerfiles. 
				\smallskip
				\item \textbf{\textit{load\_images.sh}} : Carga las imágenes preconstruidas (desde la carpeta 2). 
				\smallskip
				\item \textbf{\textit{setup-network.sh}} : Crea la red, inicia los contenedores y configura las interfaces, rutas y reglas de forwarding en el router. 
				\smallskip
				\item \textbf{\textit{test-network.sh}} : Prueba la conectividad entre los hosts a través del router. 
				\smallskip
			\end{itemize}
		
		\bigskip\bigskip
		%------------------------------------------------------------%
		
		%------------------------------------------------------------%
		\item \textbf{\textit{Reto: Unificar Imagen de Hosts }}
		
		\medskip
		\noindent Actualmente, hay dos imágenes diferentes para \textbf{\textit{host1}} y \textbf{\textit{host2}}. Tu objetivo es: 
			
		\medskip
			\begin{enumerate}
				\item Configurar un \textbf{único Dockerfile} que sirva para ambos hosts.
				\smallskip
				\item Modificar los scripts (\textbf{\textit{build\_images.sh}}, \textbf{\textit{setup-network.sh}}) para usar esta imagen.
				\smallskip
				\item Asegurarte de que la comunicación entre hosts \textbf{sigue funcionando} después del cambio.
				\smallskip
				\item (Opcional) Documentar qué cambios hiciste y por qué.
			\end{enumerate}
		
		\bigskip\bigskip
		%------------------------------------------------------------%

		
	\end{enumerate}
	
\end{document}
