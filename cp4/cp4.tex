\documentclass[12pt]{amsart}

\addtolength{\hoffset}{-2.25cm}
\addtolength{\textwidth}{4.5cm}
\addtolength{\voffset}{-2.5cm}
\addtolength{\textheight}{5cm}
\setlength{\parskip}{0pt}
\setlength{\parindent}{15pt}

\usepackage{amsthm}
\usepackage{amsmath}
\usepackage{amssymb}
\usepackage[spanish]{babel}
\usepackage[colorlinks = true, linkcolor = black, citecolor = black, final]{hyperref}

\usepackage{graphicx}
\usepackage{multicol}
\usepackage{ marvosym }
\usepackage{wasysym}
\usepackage{tikz}
\usetikzlibrary{patterns}

\newcommand{\ds}{\displaystyle}
\newcommand{\horrule}[1]{\rule{\linewidth}{#1}}

\setlength{\parindent}{0in}

\pagestyle{empty}
\begin{document}
	\hrule
	\smallskip
	\begin{center}
		{\scshape {\large Redes de Computadoras} \\
			Curso 2025-2026} \\ \smallskip
		\textbf{Clase Práctica \# 4} \\
		{\small \textbf{Tema:} Capa de Red.}
	\end{center}
	\vspace{-8px}
	\rule{\linewidth}{2pt}
	
	{\scshape Facultad de Matemática y Computación}  \hfill {\scshape Universidad de La Habana}
	
	\bigskip\bigskip
	
	
	
	\vspace{1cm}
	
	\begin{center}
		\textit{La única forma de hacer un gran trabajo es amar lo que haces.}
	\end{center}
	
	\begin{flushright}
		- Steve Jobs
	\end{flushright}
	
	\vspace{1cm}
	
	% Ejercicios
	\begin{enumerate}
		
		\thispagestyle{empty}
		
		%------------------------------------------------------------%
		\item \textbf{\textit{Introducción y Contexto }}
		
		\medskip
		\noindent
		
		\begin{enumerate}
			\item \textbf{\textit{IP (Internet Protocol):}} Permite la identificación única de cada dispositivo y define cómo se empaquetan y envían los datos (datagramas).  
			\medskip \medskip
			\item \textbf{\textit{Capa de Internet/Red:}} Transporta los datos de origen a destino a través de múltiples redes.
		\end{enumerate}
		
		\bigskip\bigskip
		%------------------------------------------------------------%

		%------------------------------------------------------------%
		\item \textbf{\textit{Direccionamiento (IPv4)}}
		
		\medskip
		\noindent
	
		\begin{enumerate}
			\item \textbf{\textit{IPv4:}} Dirección de \textbf{32 bits}  (ej., `192.168.1.1').  

			\medskip \medskip

			\item \textbf{\textit{Máscara de Subred:}} Una dirección de 32 bits que separa red y host mediante una operación lógica \textbf{AND}. 
			\medskip \medskip
			\begin{itemize}
				\item \textbf{\textit{Notación CIDR:}} Ej., `/24' equivale a `255.255.255.0'.
			\end{itemize}
		\end{enumerate}
		
		\bigskip
		\textbf{\textit{Ejercicio Práctico:}} Dada una IP y su máscara, determine la \textbf{dirección de red}.
		\medskip \medskip
			\begin{table}[h]
			    \centering
			    \begin{tabular}{|l|l|l|l|}
			        \hline
			        \textbf{Dirección IP} & \textbf{Máscara CIDR} & \textbf{Máscara Decimal} & \textbf{Dirección de Red} \\
			        \hline
			        172.16.10.50 & /16 & 255.255.0.0 & \textbf{172.16.0.0} \\
			        \hline
			        192.168.100.12 & /24 & 255.255.255.0 & \textbf{192.168.100.0} \\
			        \hline
			    \end{tabular}
			   \smallskip
			    \caption{Detalles de Direcciones IP}
			    \label{tab:detalles_direcciones_ip}
			\end{table}
		
		\bigskip\bigskip
		%------------------------------------------------------------%

		%------------------------------------------------------------%
		\item \textbf{\textit{Protocolos de Gestión y Resolución}}
		
		\medskip
		\noindent
		
		\begin{enumerate}
			\item \textbf{\textit{DHCP (Dynamic Host Configuration Protocol):}} Configuración automática (IP, Máscara, Gateway, DNS).  
			\smallskip
				\begin{itemize}
					\item Flujo \textbf{DORA:} Discovery → Offer → Request → Acknowledge.
				\end{itemize}

			\medskip \medskip

			\item \textbf{\textit{ARP (Address Resolution Protocol):}} Traduce IP → MAC. 
				\begin{itemize}
					\item Flujo: Host envía \textbf{ARP Request} (broadcast). El host destino responde con su MAC (unicast). 
				\end{itemize}

			\medskip \medskip

			\item \textbf{\textit{ICMP (Internet Control Message Protocol):}} Diagnóstico y control (ej. errores, `ping').  

		\end{enumerate}
		
		\bigskip\bigskip
		%------------------------------------------------------------%

		%------------------------------------------------------------%
		\item \textbf{\textit{Traducción y Evolución}}
		
		\medskip
		\noindent
		
		\begin{enumerate}
			\item \textbf{\textit{NAT (Network Address Translation):}} El router traduce IPs privadas en una pública.  
			\smallskip
				\begin{itemize}
					\item Caso común: \textbf{PAT (Port Address Translation)}, usa puertos para distinguir sesiones.  
				\end{itemize}

			\medskip \medskip

			\item\textbf{\textit{IPv6:}} Evolución de IP.  
				\medskip
				\begin{itemize}
					\item \textbf{128 bits}, notación hexadecimal (ej. `2001:db8:acad:1::1').  
					\medskip
					\item Ventajas: Espacio ilimitado, autoconfiguración (SLAAC), seguridad (IPsec).  
				\end{itemize}
			
			\medskip \medskip			

			\item \textbf{\textit{Tabla Comparativa}}
			\medskip \medskip
			
			\begin{table}[h]
			    \centering
			    \begin{tabular}{|l|l|l|}
			        \hline
			        \textbf{Característica} & \textbf{IPv4} & \textbf{IPv6} \\
			        \hline
			        Tamaño Dirección & 32 bits & 128 bits \\
			        \hline
			        Notación & Decimal (.) & Hexadecimal (:) \\
			        \hline
			        Escasez & Usa NAT & No requiere NAT \\
			        \hline
			    \end{tabular}
			    \smallskip
			    \caption{Comparación entre IPv4 e IPv6}
			    \label{tab:comparacion_ipv4_ipv6}
			\end{table}

		\end{enumerate}
		
		\bigskip\bigskip
		%------------------------------------------------------------%

		%------------------------------------------------------------%
		\item \textbf{\textit{Conclusión}}
		
		\medskip \medskip
		\noindent
		
		\begin{table}[h]
		    \centering
		    \begin{tabular}{|l|l|l|}
		        \hline
		        \textbf{Protocolo} & \textbf{Función Principal} & \textbf{Ejemplo de Uso} \\
		        \hline
		        IP & Direccionamiento y Enrutamiento. & Presente en cualquier dirección. \\
		        \hline
		        Máscara & Separar Red y Host. & Configuración de LAN. \\
		        \hline
		        DHCP & Configuración automática. & Conexión a Wi-Fi. \\
		        \hline
		        ARP & Mapear IP $\to$ MAC. & Comunicación dentro de LAN. \\
		        \hline
		        ICMP & Control y diagnóstico. & Comando \texttt{ping}. \\
		        \hline
		        NAT & Traducir IP privadas a una pública. & Routers domésticos. \\
		        \hline
		    \end{tabular}
		   \smallskip
		    \caption{Descripción de Protocolos de Red}
		    \label{tab:protocolos_red}
		\end{table}
		
		\bigskip\bigskip
		%------------------------------------------------------------%


	\end{enumerate}
	
\end{document}
